\documentclass[12pt]{article}
\usepackage[margin=1.5cm]{geometry}
\usepackage{hyperref, setspace, multicol}
\title{Midterm Report}
\author{Dylan Ahearn, Aubrey Conners, Christopher Siems, Samuel Szymanski}
\date{2024 March 18}
\begin{document}
  \parindent=0pt
  \parskip=6pt
  \maketitle
  \onehalfspacing
  \section*{System Overview}
    Our project is built around three main processes. These parts are the testing of our set of Pac-Man algorithms, the collection of data about the performance of each Pac-Man algorithm, and the processing of that data. The first process is having each algorithm run many times through a standard game of Pac-Man. The second process is collecting data from each algorithm about their performance for that individual run and storing that data in a dataset. And the final process is using the dataset we have developed to create graphs of the performance of each algorithm and extract key data points from that stored performance.
  \section*{System Architecture}
    There are four components that will make up the whole project.  
    \subsection*{The Game}
      Probably most obvious component is the Pac-Man game itself. The Pac-Man version that we are using is a faithful Pac-Man implementation built in Python using Pygame that was forked and modified for our purposes from an open-source Pygame Pac-Man implementation built by Sebastian Rohaut, \href{https://github.com/usawa/pypacman}{usawa/pypacman} on GitHub. This implementation is a very high-quality clone of Pac-Man that was easy to set up and begin modifying for our purposes. The primary change we have made to Rohaut's Pac-Man is editing the code to allow instances of the Pac-Man object, that represents Pac-Man himself, to be constructed with an algorithm passed as an argument in the constructor. 
    \subsection*{The Algorithms}
      This component is really a set of components. This is the set of algorithms that will be used to control Pac-Man. Each algorithm will be contained within a class representing that algorithm and containing all its logic. Instances of these algorithm classes will be instantiated and passed as arguments to the Pac-Man constructor to set that algorithm as the controller for that instance of Pac-Man. Some, but not all, ideas we are moving forward with to implement in our algorithms are:
      \begin{itemize}
        \itemsep-4pt
          \item Randomness. To be used as a baseline of comparison. How well does Pac-Man perform when he is moving randomly? (Spoiler: Not well!)
          \item Predictable patterns. Algorithms that play the game in predictable ways.
          \item Choosing points. Algorithms that select points in the maze to move to, based on some criteria, and move to them using some pathfinding algorithm.
          \item Breadth First Search. Will serve two purposes in our project, to serve as a point-to-point pathfinding algorithm and a modified version will serve as a complete Pac-Man algorithm.
          \item A*. One of the algorithms suitable for point-to-point pathfinding.
          \item Dijkstras algorithm . One of the algorithms suitable for point-to-point pathfinding.
      \end{itemize}
    \subsection*{Data Collection}
      Data about the performance of each algorithm will be collected at the end of each algorithm's attempt at playing Pac-Man. This data will be collected into a Comma Separated Values (CSV) file or multiple CSV files for later reference using Python's Pandas library. Some of the data points we intend to track include:
      \begin{itemize}
        \itemsep-4pt
        \item Success. Did Pac-Man win?
        \item Lives lost. The number of lives Pac-Man lost
        \item Points scored. The number of points Pac-Man scored
        \item Pacgums collected. The number of Pacgums (the dots) Pac-Man collected.
      \end{itemize}
    \subsection*{Data Processing}
      The aforementioned data will later be processed into a smaller dataset of key performance metrics for each algorithm and a set of graphs meant to graphically display the performance of each algorithm and compare said performance. These actions will be completed using the Python libraries Pandas, Scipy, and Pyplot from Matplotlib. Some performance metrics we intend to collect from the data include:
      \begin{multicols}{2}
        \begin{itemize}
          \itemsep-4pt
          \item Means
          \item Medians
          \item Modes
          \item Ranges
          \item Standard Deviations 
          \item Quartiles 
        \end{itemize}
      \end{multicols}
      And some graphs we intend to produce include: 
      \begin{itemize}
        \itemsep-4pt
        \item Histograms
        \item Scatter plots
        \item Bar charts
      \end{itemize}
  \section*{Component Interaction}
    There are three interactions between the components explained in the prior section.
    \subsection*{Game and Algorithm}
      The game interacts with the algorithms. This is handled by instantiating instances of the algorithms and passing them into the \texttt{main()} function in the \texttt{pypacman.py} file. From there they are passed into the constructor for the game itself and the instance of the game is assigned to a variable inside the algorithm object, this gives the algorithm access to the game's state. The algorithm is then passed into Pac-Man's constructor and assigned as Pac-Man's algorithm for that run of the game. The algorithm's \texttt{setup()} method is then called to perform any calculations necessary before the game begins. From that point, the algorithm's \texttt{get\_dir()} method is called after each move to get the direction that the algorithm would have Pac-Man face for the next move.
    \subsection*{Game and Datasets}
      The game interacts with the datasets. After each run of the game, the game will spit out information about the performance of Pac-Man in during that game. This information will be compiled into a dataset in the form of a Pandas DataFrame and then appended to the end of a CSV file tracking the statistics of all algorithm runs thus far.
    \subsection*{Datasets and Data Processor}
      The processing script interacts with the dataset to produce a smaller dataset of the key values enumerated in the prior section for each algorithm and produces a set of graphs depicting information about the performance of each algorithm and graphs comparing the performance of each algorithm. 
  \vfill
  Typeset with \LaTeX.
\end{document}